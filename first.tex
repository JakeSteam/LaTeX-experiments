% Need "book" to handle chapters
\documentclass[12pt]{book} 
\usepackage{graphicx} % Imports graphicx package
\graphicspath{{images/}} % Configures graphicx
\title{A little LaTeX document}
\author{Jake Lee}
\date{Oct 2023}
\begin{document}
\maketitle

\chapter{First tests}
Wowsa, the title gets generated just from the metadata. Cool.

\section{Basic formatting}
We can do \textbf{bold text}, \underline{underlined text}, and \textit{italic text}.

\section{Emphasis}
"Emphasis" is useful when nested inside bold / italic text:

Some words \emph{and emphasis} within a sentence.

\textit{Some words \emph{and emphasis} within a sentence.}

\textbf{Some words \emph{and emphasis} within a sentence.}

\section{Images}
% Note: Usually omit extension, and we defined path earlier
\includegraphics{skull}

% Note: Interestingly, the numbering is handled automatically
\begin{figure}
    \centering
    \includegraphics{skull}
    \caption{A cute skull!}
    \label{fig:skull}
\end{figure}

% Note: Defining width as a fraction of text makes sense
\begin{figure}
    \centering
    \includegraphics[width=0.5\textwidth]{skull}
    \caption{A cute skull!}
    \label{fig:skull2}
\end{figure}

\section{Lists}

\begin{itemize}
    \item Here's one list item 
    \item Here's another
    \item Basically the same as ul and li
\end{itemize}

\begin{enumerate}
    \item Ordered this time 
    \item Very cool
    \item Same as ol and li
\end{enumerate}

\section{Maths} 
\subsection{Inline maths (3 variants)}
\begin{math}E=mc^2\end{math} is typeset in a paragraph using inline math mode---as is $E=mc^2$, and so too is \(E=mc^2\).

\subsection{Display maths (2 variants)}
The mass-energy equivalence is described by the famous equation \[ E=mc^2 \] discovered in 1905 by Albert Einstein. 

The mass-energy equivalence is described by the famous equation
\begin{equation}
    E=mc^2
\end{equation}
discovered in 1905 by Albert Einstein. 

\subsection{Detailed maths}

Subscripts and superscripts:

\[ T^{i_1 i_2 \dots i_p}_{j_1 j_2 \dots j_q} = T(x^{i_1},\dots,x^{i_p},e_{j_1},\dots,e_{j_q}) \]

Random symbols:

\[ \omega \delta \Omega \Delta \sin \cos \tan \]

\end{document}